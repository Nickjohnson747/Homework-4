% MathMode: plain, MathRender: svg, MathDpi: 300, MathEmbedLimit: 524288, MathScale: 105, MathBaseline: 0, MathDocClass: [10pt]book, MathImgDir: math, MathLatex: xelatex, MathSvgUseFonts: False, MathSvgSharePaths: True, MathSvgPrecision: 3, Dvisvg: dvisvgm
\documentclass[10pt]{book}
% generated by Madoko, version 1.0.3
%mdk-data-line={1}
\newcommand\mdmathmode{plain}
\newcommand\mdmathrender{svg}
\usepackage[heading-base={2},section-num={false},bib-label={true}]{madoko2}


\begin{document}


\begin{mdSnippets}
%mdk-begin-mathdefs
%mdk-data-line={8}
\newcommand{\jacobi}[2]{\ensuremath{\left(\frac{#1}{#2}\right)}}

%mdk-data-line={17}
\begin{mdDisplaySnippet}[c30d761438fd242b7366fbabe3597afd]%mdk
\[%mdk-data-line={18}
\frac {n!} / {(n-r!)}
\]%mdk
\end{mdDisplaySnippet}%mdk
%mdk-data-line={21}
\begin{mdInlineSnippet}[bf923beca9131efc991b7ae22bff45ec]%mdk
$C(15,5) = 3003$\end{mdInlineSnippet}%mdk
%mdk-data-line={24}
\begin{mdInlineSnippet}[775aa9f7cc0de1e014b3e4955b2f4fd5]%mdk
$C(15,3) * C(10,2) = 20,475$\end{mdInlineSnippet}%mdk
%mdk-data-line={60}

\end{mdSnippets}

\end{document}
